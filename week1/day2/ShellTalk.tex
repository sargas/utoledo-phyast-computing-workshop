\documentclass[xcolor={usenames,x11names}]{beamer}

\usepackage{default}
\usepackage{lmodern}
\usepackage{booktabs}
\usepackage{attrib}
\usepackage{listings}


\usetheme{Boadilla}
\usecolortheme{whale}

\title[Linux/UNIX Shell]{Making the Linux/UNIX Shell Productive}
\author{Joseph Booker}
\institute[UToledo]{Department of Physics and Astronomy \\ University of Toledo}

\setbeamertemplate{section page}
{%
    \begin{centering}
    \begin{beamercolorbox}[sep=12pt,center]{part title}
    \usebeamerfont{section title}\insertsection\par
    \end{beamercolorbox}
    \end{centering}
}


\begin{document}

\frame{\titlepage}

\begin{frame}{Layout of Today}
	\begin{itemize}
		\item Overview of different shells used in academia
		\item Manipulating I/O (pipes and redirections)
		\item Environmental Variables
		\item Creating aliases
		\item Startup files
		\item Useful tricks: Wildcards, History
	\end{itemize}
\end{frame}

\section{Overview of Shells}
\frame{\sectionpage}

\begin{frame}{History of the Shell(s)}
	\centering
	\begin{tabular}{l||l|r}
		\toprule
		Late 1970s & Bourne Shell (sh) & C Shell (csh) \\
		\midrule
		\pause Early 1980s & Korn Shell (ksh) & \alert<5>{tcsh} \\
		\midrule
		\pause Late 1980s & \alert<5>{Bourne-Again Shell (bash)} & \\
		\midrule
		\pause Newer & Z Shell, ash, dash, fish& \\
		\bottomrule
	\end{tabular}
\end{frame}

\begin{frame}{C Shell Considered Harmful}
	\begin{itemize}
		\item Somewhat C-like, but with an ad hoc parser that makes programming difficult
		\pause\item Confusing quoting rules --- especially if multi-line or using special symbols
		\pause\item Limited manipulation of input/output
	\end{itemize}
	\pause\begin{quote}
		When I started doing this stuff with Unix, I wasn't a very good programmer.
		\attrib{Bill Joy on the ad hoc parser (2009)}
	\end{quote}
\end{frame}

\section{Startup Files}
\frame{\sectionpage}

\begin{frame}{Ways to start a shell}
	\begin{description}
		\pause\item[Interactively] Are you running the shell, or is a script?
		\pause\item[Login shell] Is this the first shell you started when logging in?
	\end{description}
\end{frame}

\begin{frame}{A Sane Way to Start Shells (tcsh)}
	\begin{description}
		\pause\item[All Shells] reads \path{~/.tcshrc} if it exists, else \path{~/.cshrc}
		\pause\item[Login Shells] reads \path{~/.login} if it exists
	\end{description}
\end{frame}

\begin{frame}{A Fine-tuned Way to Run Shells}
	\begin{description}
		\pause\item[Login Shells] checks for \path{~/.bash_profile}, \path{~/.bash_login}, and \path{~/.profile}: Loads the first one which exists.
		\pause\item[Interactive, Non-Login Shells] reads \path{~/.bashrc} if it exists
	\end{description}
\end{frame}

\begin{frame}[fragile]{Simplify life by always using \path{~/.bashrc}}
	\begin{lstlisting}[language=bash, caption={Put this in \textasciitilde/.bash\_profile}]
	if [ -f ~/.bashrc ];
		then . ~/.bashrc;
	fi
	\end{lstlisting}
\end{frame}

\end{document}
